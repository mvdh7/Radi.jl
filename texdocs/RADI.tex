\documentclass[a4paper]{article}
\usepackage{siunitx}
\usepackage{natbib}
\bibliographystyle{abbrvnat}
\usepackage[version=4]{mhchem}
\usepackage[dvipsnames]{xcolor}
\usepackage{amsmath}

\newcommand{\D}[1]{\mathrm{d}#1}
\newcommand{\code}[1]{\texttt{#1}}
\newcommand{\uM}{\si{\micro}\mathrm{M}}
\newcommand{\molmmm}{\mathrm{mol}\cdot\mathrm{m}^{-3}}
\newcommand{\degC}{\si{\degree}\mathrm{C}}
\newcommand{\wtf}[1]{\textcolor{Cerulean}{\textbf{#1}}}
\newcommand{\codenote}[1]{\textcolor{Orange}{\textbf{#1}}}

\begin{document}

\title{RADI.jl}
\author{Olivier Sulpis and Matthew P. Humphreys}

\maketitle

\begin{abstract}
RADI.jl is a Julia implementation of RADI: the 1-D Reaction-Advection-Diffusion-Irrigation Diagenetic Sediment Module. Here, we define the variables and the equations used in this implementation. So far, only one solute (dissolved oxygen) and one solid (particulate organic carbon) are included and documented here.
\end{abstract}

\section{Parameters}

\subsection{Time}

Time units are always in years.
\begin{itemize}
  \item $T$ ($\code{stoptime}$ in a) is the total time that the model runs for.
  \item $\D{t}$ ($\code{interval}$ in a) is the time resolution (i.e. the interval between each timestep).
  \item $t$ ($\code{timesteps}$ in a) refers to the array of modelled timepoints.
\end{itemize}
The model therefore runs from time $0$ to $T$ in intervals of $\D{t}$.

\subsection{Sediment column}

\subsubsection{Structure}

Depth units are always in metres.
\begin{itemize}
  \item $Z$ ($\code{z\_max}$ in m) is the total height of the sediment column being modelled.
  \item $\D{z}$ ($\code{z\_res}$ in m) is the depth resolution (i.e. the height of each model layer).
  \item $z$ ($\code{depths}$ in m) refers to the array of modelled depths within the sediment.
\end{itemize}
The model layers are therefore at depths within the sediment from $0$ to $Z$ in increments of $\D{z}$, where $0$ represents the interface between the surface sediment and overlying seawater.

\subsubsection{Overlying water}

Properties of the overlying water can be set by the user:

\begin{itemize}
  \item $[\ce{O2}]_w$ ($\code{oxy\_w}$ in $\molmmm$) is the seawater dissolved oxygen concentration.
  \item $F_c$ ($\code{Foc}$ in mol$\cdot$m$^{-1}\cdot$a$^{-1}$) is the flux of particulate organic carbon arriving at the seafloor.
  \item $\delta$ ($\code{dbl}$ in m) is the diffusive boundary layer thickness.
  \item $T_w$ ($\code{T}$ in $\degC$) is the water temperature.
\end{itemize}
\codenote{Note that $T$ is also used for the total model runtime...}

\subsubsection{Sediment properties}

In the current set-up of RADI.jl, the porosity parameters $\phi_\infty$ and $\phi_0$ are set to 0.74 and 0.85 respectively ($\code{phiInf}$ and $\code{phi0}$, both dimensionless), and $\beta = 33$ m$^{-1}$ ($\code{beta}$). These values were obtained by fitting real data from station 7, mooring 3 of cruise NBP98-2 \citep{sayles_benthic_2001}. \wtf{Should these ($\phi_0$, $\phi_\infty$ and $\beta$) be inputs that the user can easily adjust?}

The porosity profile ($\phi$, $\code{phi}$, dimensionless) is parameterised following \citet{boudreau_method--lines_1996}:
\begin{equation}\label{phi}
\phi = \phi_\infty + (\phi_0 - \phi_\infty) \exp(-\beta z)
\end{equation}
The corresponding ``solid porosity'' ($\phi_s$, $\code{phiS}$, dimensionless) is:
\begin{equation}\label{phiS}
\phi_s = 1 - \phi
\end{equation}
RADI.jl also uses the related convenience variable $\code{phiS\_phi} = \phi_s/\phi$.

Following \citet{boudreau_diffusive_1996}, the sediment tortuosity ($\theta$, $\code{tort}$, dimensionless) is:
\begin{equation}\label{tort}
\theta = \sqrt{1 - 2 \log \phi}
\end{equation}
RADI.jl also uses the related convenience variable $\code{tort2} = \theta^2$.

Following \citet{archer_model_2002}, the surface sediment bioturbation coefficient $B_0$ ($\code{D\_bio\_0}$ in m$^2\cdot$a$^{-1}$) is:
\begin{equation}\label{D_bio_0}
B_0 = (0.0232 \cdot 10^{-4}) (F_c \cdot 10^2)^{\,0.85}
\end{equation}
\wtf{Eq. \eqref{D_bio_0}: where do the $10^{-4}$ and $10^2$ multipliers come from? These are in the code but not in the documentation.}

The bioturbation coefficient propagates down through the sediment as $B_z$ ($\code{D\_bio}$ in m$^2\cdot$a$^{-1}$):
\begin{equation}\label{B_z}
B_z = B_0 \exp(-z/\lambda_b) [\ce{O2}]_w / ([\ce{O2}]_w + 0.02 \, \molmmm)
\end{equation}
where $\lambda_b$ ($\code{lambda\_b}$ in m) is the characteristic depth of 0.08 m, following \citet{sayles_benthic_2001}. \wtf{Where does the magic 0.02 number come from?}

The rate constant for organic matter degradation ($k_z$, $\code{krefractory}$ in a$^{-1}$) is \citep{archer_model_2002}:
\begin{equation}\label{k_poc_degradation}
k_z = 80.25 \, B_0 \exp(-z)
\end{equation}

\section{Variables}

\subsection{Porewater solutes}

Within the sediment porewaters:
\begin{itemize}
  \item $[\ce{O2}]$ (\code{oxy} in $\molmmm$) is the dissolved oxygen concentration.
\end{itemize}

\subsection{Solids}

Within the sediment itself:
\begin{itemize}
  \item $[$POC$]$ (\code{poc} in $\molmmm$) is the particulate organic carbon concentration.
\end{itemize}

\section{Master equation}

For each modelled variable $v$ at time $t$ and depth $z$:
\begin{equation}
v_{(t+\D{t}), z} = v_{t,z} + [R(v_{t,z}) + A(v_{t,z}) + D(v_{t,z}) + I(v_{t,z})] \cdot \D{t}
\end{equation}
where:
\begin{itemize}
  \item $R(v_{t,z})$ quantifies the rate of change of $v_{t,z}$ due to \textbf{reactions}.
  \item $A(v_{t,z})$ quantifies the rate of change of $v_{t,z}$ due to \textbf{advection}.
  \item $D(v_{t,z})$ quantifies the rate of change of $v_{t,z}$ due to \textbf{diffusion}.
  \item $I(v_{t,z})$ quantifies the rate of change of $v_{t,z}$ due to \textbf{irrigation}.
\end{itemize}

\section{Reaction}

Biogeochemical reactions operate on solutes and solids throughout the entire sediment column, including the very top and bottom layers. $R(v_t)$ is the net rate at which $v$ is being consumed (negative $R$) or created (positive $R$) by these reactions.

\subsection{Organic matter degradation}

Organic matter degradation affects dissolved oxygen and particulate organic carbon:
\begin{equation}
R(\mathrm{POC}) = -k_z [\mathrm{POC}]
\end{equation}
\begin{equation}\label{r_O2}
R(\ce{O2}) = r(\mathrm{POC}) \cdot \phi_s / \phi
\end{equation}
where the rate constant $k_z$ was defined in Eq. \eqref{k_poc_degradation}, and porosity coefficients $\phi$ and $\phi_s$ in Eqs. \eqref{phi} and \eqref{phiS} respectively. \wtf{Eq. \eqref{r_O2}: should there not be a photosynthetic quotient (C:O$_2$ ratio) in here?}

\section{Advection}

\wtf{By "advection" do we mean things literally moving around within a static sediment (i.e. porewater is advecting relative to the centre of the Earth), or is it that things to move relatively speaking because the sediment surface moves up as new material settles onto it (i.e. porewater is static relative to the centre of the Earth, but advecting relative to the sediment-water interface)?}

Advection is modelled differently (1) at the sediment-water interface (i.e. where $z = 0$), (2) within the sediment ($0 < z < Z$), and (3) at the bottom of the sediment ($z = Z$).

\subsection{Advection at the sediment-water interface}\label{sx:adv_swi}

At sediment depth $z = 0$ only, for solutes (e.g. $\ce{O2}$):
\begin{equation}
A(v_0) = u_0 (v_w - v_0) \theta^2 / \delta
\end{equation}
and for solids (e.g. POC):
\begin{equation}
A(v_0) = w_0 (F_c - \phi_s w_0 v_0) / (B_0 \phi_s)
\end{equation}

\subsection{Advection within the sediment}

At sediment depth $z$, where $0 < z < Z$, for solutes (e.g. $\ce{O2}$):
\begin{equation}
A(v_z) = -[u_z - (\theta^2 \D{\phi} / \phi - \D{\{\theta^2\}}) d(v_z) / \theta^2] [v_{(z+\D{z})} - v_{(z-\D{z})}] / 2 \D{z}
\end{equation}
where $d(v_z)$ is the relevant diffusion coefficient given in section \ref{sx:diff}.

For solids (e.g. POC):
\begin{equation}
\begin{split}
A(v_z) = & -[(1 - \sigma_z) v_{(z+\D{z})} + 2 \sigma_z v_z - (1 + \sigma_z) v_{(z-\D{z})}] \cdot \\
& [w_z - \D{B_z} - \D{\phi_s} B_z / \phi_s] / 2 \D{z}
\end{split}
\end{equation}
where $\sigma_z$ ($\code{sigma}$) is given by:
\begin{equation}
\sigma_z = 1 / \mathrm{tanh}(P_{e_h}) - 1 / P_{e_h}
\end{equation}
in which $P_{e_h}$ ($\code{Peh}$) is:
\begin{equation}
P_{e_h} = w_z \D{z} / (2 B_z)
\end{equation}

\subsection{Advection at the bottom of the sediment}\label{sx:adv_bot}

Solutes (e.g. \ce{O2}) are not affected by advection at the bottom of the sediment.

At sediment depth $z = Z$ only, and for solids only (e.g. POC):
\begin{equation}
A(v_Z) = -(w_Z - \D{B_Z} - \D{\phi_s} B_Z / \phi_s) (v_Z - v_{(Z-\D{z})}) \sigma_Z / \D{z}
\end{equation}

\section{Diffusion}\label{sx:diff}

Diffusion is modelled differently (1) at the sediment-water interface (i.e. where $z = 0$), (2) within the sediment ($0 < z < Z$), and (3) at the bottom of the sediment ($z = Z$).

Diffusion is controlled by each variable's diffusion coefficient (generically $d(v)$, $\code{D\_var}$ in m$^2\cdot$a$^{-1}$).

For dissolved oxygen:
\begin{equation}
d(\ce{O2}) = (0.034862 + 0.001409 T_w) / \theta^2
\end{equation}
where the temperature function is from \citet{li_diffusion_1974}, and $\theta$ is the tortuosity from Eq. \eqref{tort}.

For particulate organic carbon:
\begin{equation}
d(\mathrm{POC}) = B_z
\end{equation}
where $B_z$ was defined in Eq. \eqref{B_z}.

\subsection{Diffusion at the sediment-water interface}\label{sx:diff_swi}

At sediment depth $z = 0$ only, for solutes (e.g. \ce{O2}):
\begin{equation}\label{Dv0}
D(v_0) = 2 d(v_z) \cdot [v_{\D{z}} - v_0 + \theta_0^2 \D{z} (v_w - v_0) / \delta] / (\D{z})^2
\end{equation}
where $\theta_0$ is the tortuosity at the sediment surface from Eq. \eqref{tort}. RADI.jl uses the convenience variable $\code{TR} = 2 \theta_0^2 \D{z} / \delta$ to simplify Eq. \eqref{Dv0}.

\wtf{Is the factor of 2 in Eq. \eqref{Dv0} above incorrect? It doubles the upwards diffusive flux relative to that within the sediment. See also similar note in section \ref{sx:diff_bot} below.}

For solids (e.g. POC):
\begin{equation}
D(v_0) = 2 d(v_z) \cdot [v_{\D{z}} - v_0 + (F_c - \phi_s w_0 v_0) / (B_0 \phi_s)] / (\D{z})^2
\end{equation}

\subsection{Diffusion within the sediment}

At sediment depth $z$, where $0 < z < Z$, for both solutes and solids:
\begin{equation}
D(v_z) = d(v_z) \cdot (v_{(z+\D{z})} - 2 v_z + v_{(z-\D{z})}) / (\D{z})^2
\end{equation}
where $d(v_z)$ is the relevant diffusion coefficient.

\subsection{Diffusion at the bottom of the sediment}\label{sx:diff_bot}

At sediment depth $z = Z$ only, for both solutes and solids:
\begin{equation}
D(v_Z) = 2 d(v_z) \cdot (-v_Z + v_{(Z-\D{z})}) / (\D{z})^2
\end{equation}
\wtf{Is the factor of 2 here incorrect? Without it, this equation would be the same as for `within the sediment', but with no diffusive flux coming up from below. With it, this assumes that whatever diffusive flux comes from above, also comes up from below (i.e. the flux from above is doubled). See also similar note in section \ref{sx:diff_swi} above.}

\section{Irrigation}

Irrigation only affects the solutes (e.g. \ce{O2}), not the solids (e.g. POC). Its effect is consistent throughout the sediment:
\begin{equation}
I(v_z) = \alpha_z \, (v_w - v_z)
\end{equation}
The coefficient $\alpha_z$ ($\code{alpha}$ in a$^{-1}$) at the sediment-water interface (i.e. $\alpha_0$, $\code{alpha\_0}$) is, following \citet{archer_model_2002}:
\begin{equation}
\begin{split}
\alpha_0 = & 11 \Bigg[ \mathrm{atan} \Bigg( \frac{5 F_c \cdot 10^2 - 400}{400 \pi} \Bigg) + 0.5 \Bigg] - 0.9 \; + \\
& \frac{20 [\ce{O2}]_w}{[\ce{O2}]_w + 0.01} \cdot \frac{F_c \cdot 10^2}{F_c \cdot 10^2 + 30} \cdot \exp \Bigg(\frac{-[\ce{O2}]_w}{0.01}\Bigg)
\end{split}
\end{equation}
\wtf{Where do all these magic numbers come from?}

Within the sediment itself:
\begin{equation}
\alpha_z = \alpha_0 \exp [-(z/\lambda_i)^2]
\end{equation}
where $\lambda_i = 0.05$ m ($\code{lambda\_i}$) is the \wtf{somewhat mysterious} characteristic depth of \citet{archer_model_2002}.

\bibliography{RADI}

\end{document}
