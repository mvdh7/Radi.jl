\documentclass[a4paper]{article}
\usepackage{siunitx}
\usepackage{natbib}
\bibliographystyle{abbrvnat}
\usepackage[version=4]{mhchem}
\usepackage[dvipsnames]{xcolor}

\newcommand{\D}[1]{\mathrm{d}#1}
\newcommand{\code}[1]{\texttt{#1}}
%\newcommand{\conc}[1]{[\code{#1}]}
\newcommand{\uM}{\si{\micro}\mathrm{M}}
\newcommand{\molmmm}{\mathrm{mol}\cdot\mathrm{m}^{-3}}
\newcommand{\wtf}[1]{\textcolor{Aquamarine}{\textbf{#1}}}

\begin{document}

\title{RADI.jl}
\author{Olivier Sulpis and Matthew P. Humphreys}

\maketitle

\begin{abstract}
RADI.jl is a Julia implementation of RADI: the 1-D Reaction-Advection-Diffusion-Irrigation Diagenetic Sediment Module. Here, we define the variables and the equations used in this implementation. So far, only one solute (dissolved oxygen) and one solid (particulate organic carbon) are included and documented here.
\end{abstract}

% \section{Variables}
%
% \subsection{Solutes}
%
% \subsection{Solids}
%
% \section{Processes}
%
% \subsection{Solutes}
%
% \subsection{Solids}

\section{Parameters}

\subsection{Time}

Time units are always in years.
\begin{itemize}
  \item $T$ ($\code{stoptime}$ in a) is the total time that the model runs for.
  \item $\D{t}$ ($\code{interval}$ in a) is the time resolution (i.e. the interval between each timestep).
  \item $t$ ($\code{timesteps}$ in a) refers to the array of modelled timepoints.
\end{itemize}
The model therefore runs from time $0$ to $T$ in intervals of $\D{t}$.

\subsection{Sediment column}

\subsubsection{Structure}

Depth units are always in metres.
\begin{itemize}
  \item $Z$ ($\code{z\_max}$ in m) is the total thickness of the sediment column being modelled.
  \item $\D{z}$ ($\code{z\_res}$ in m) is the depth resolution (i.e. the height of each model layer).
  \item $z$ ($\code{depths}$ in m) refers to the array of modelled depths within the sediment.
\end{itemize}
The model layers are therefore at depths within the sediment from $0$ to $Z$ in increments of $\D{z}$, where $0$ represents the interface between the surface sediment and overlying seawater.

\subsubsection{Overlying water}

Properties of the overlying water can be changed by the user each time RADI.jl runs.

\begin{itemize}
  \item $[\ce{O2}]_w$ ($\code{oxy\_w}$ in $\molmmm$) is the seawater dissolved oxygen concentration.
  \item $F_c$ ($\code{Foc}$ in mol$\cdot$m$^{-1}\cdot$a$^{-1}$) is the flux of particulate organic carbon arriving at the seafloor.
\end{itemize}

\subsubsection{Sediment properties}

The depth-varying porosity ($\phi$, $\code{phi}$, dimensionless) is parameterised following \citet{boudreau_method--lines_1996}:
\begin{equation}\label{phi}
\phi = \phi_\infty + (\phi_0 - \phi_\infty) \exp(-\beta z)
\end{equation}
where $\phi_\infty = 0.74$, $\phi_0 = 0.85$ ($\code{phiInf}$ and $\code{phi0}$ respectively, both dimensionless) and $\beta = 33$ m$^{-1}$ ($\code{beta}$) \citep{boudreau_method--lines_1996}. The corresponding ``solid porosity'' ($\phi_s$, $\code{phiS}$, dimensionless) is:
\begin{equation}\label{phiS}
\phi_s = 1 - \phi
\end{equation}
RADI.jl also creates the convenience variable $\code{phiS\_phi} = \phi_s/\phi$.

Following \citet{archer_model_2002}, the surface sediment bioturbation coefficient $B_0$ ($\code{D\_bio\_0}$ in m$^2\cdot$a$^{-1}$) is:
\begin{equation}\label{D_bio_0}
B_0 = (0.0232 \cdot 10^{-4}) (F_c \cdot 10^2)^{\,0.85}
\end{equation}
\wtf{Eq. \eqref{D_bio_0}: where do the $10^{-4}$ and $10^2$ multipliers come from?}

The bioturbation coefficient propagates down through the sediment as $B_z$ ($\code{D\_bio}$ in m$^2\cdot$a$^{-1}$):
\begin{equation}
B_z = B_0 \exp(-z/\lambda_b) [\ce{O2}]_w / [[\ce{O2}]_w + 0.02 \, \molmmm]
\end{equation}
where $\lambda_b$ ($\code{lambda\_b}$ in m) is the characteristic depth of 0.08 m, following \citet{sayles_benthic_2001}.

The rate constant for organic matter degradation ($k_z$, $\code{krefractory}$ in a$^{-1}$) is \citep{archer_model_2002}:
\begin{equation}\label{k_poc_degradation}
k_z = 80.25 \, B_0 \exp(-z)
\end{equation}

\section{Variables}

\subsection{Porewater solutes}

Within the sediment porewaters:
\begin{itemize}
  \item $[\ce{O2}]$ (\code{oxy} in $\molmmm$) is the dissolved oxygen concentration.
\end{itemize}

\subsection{Solids}

Within the sediment itself:
\begin{itemize}
  \item $[$POC$]$ (\code{poc} in $\molmmm$) is the particulate organic carbon concentration.
\end{itemize}

\section{Master equation}

For each modelled variable $v$:
\begin{equation}
v_{t+1} = v_t + R(v_t) + A(v_t) + D(v_t) + I(v_t)
\end{equation}
where:
\begin{itemize}
  \item $v_t$ is the concentration of the variable $v$ at timestep $t$ at a specific depth in the sediment ($z$).
  \item $R(v_t)$ quantifies the effect of \textbf{reactions} on $v$ from $t$ to $t+1$.
  \item $A(v_t)$ quantifies the effect of \textbf{advection} on $v$ from $t$ to $t+1$.
  \item $D(v_t)$ quantifies the effect of \textbf{diffusion} on $v$ from $t$ to $t+1$.
  \item $I(v_t)$ quantifies the effect of \textbf{irrigation} on $v$ from $t$ to $t+1$.
\end{itemize}

\section{Reaction}

Reaction processes operate on the entire sediment column, including the very top and bottom layers. Biogeochemical reactions for both solutes and solids are modelled as:
\begin{equation}
R(v_t) = r(v_t) \, \D{t}
\end{equation}
where $r(v_t)$ is the net rate at which $v$ is being consumed (negative $r(v_t)$) or created (positive $r(v_t)$) by biogeochemical reactions.

\subsection{Organic matter degradation}

Organic matter degradation affects dissolved oxygen and particulate organic carbon:
\begin{equation}
r(\mathrm{POC}) = -k_z [\mathrm{POC}]
\end{equation}
\begin{equation}\label{r_O2}
r(\ce{O2}) = r(\mathrm{POC}) \cdot \phi_s / \phi
\end{equation}
where the rate constant $k_z$ was defined in Eq. \eqref{k_poc_degradation}, and porosity coefficients $\phi$ and $\phi_s$ in Eqs. \eqref{phi} and \eqref{phiS} respectively. \wtf{Eq. \eqref{r_O2}: should there not be a photosynthetic quotient (C:O$_2$ ratio) in here?}

\section{Advection}

\section{Diffusion}

Diffusion is handled separately (1) at the sediment-water interface (i.e. where $z = 0$), (2) within the sediment ($0 < z < Z$), and (3) at the bottom of the sediment ($z = Z$).

Diffusion is controlled by each variable's diffusion coefficient (generically $d(v)$, $\code{D\_var}$ in m$^2\cdot$a$^{-1}$). For dissolved oxygen:
\begin{equation}
d(\ce{O2}) = (0.034862 + 0.001409 T) / \theta^2
\end{equation}
where the temperature function is from 

\subsection{Diffusion at the sediment-water interface}



\subsection{Diffusion within the sediment}

At depth $z$:

\begin{equation}
D(v_z) = d(v_z) \cdot (v_{z+1} - 2 v_z + v_{z-1}) \cdot \D{t} / (\D{z})^2
\end{equation}
where $d(v_z)$ is the relevant diffusion coefficient.

\subsection{Diffusion at the bottom of the sediment}



\section{Irrigation}

\bibliography{RADI}

\end{document}
